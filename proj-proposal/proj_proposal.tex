\documentclass[a4paper]{report}
\usepackage{times}
\usepackage[utf8]{inputenc}
\usepackage[pdftex]{graphicx}
\usepackage{multicol}

% to customize headers and footers
\usepackage{fancyhdr}
\pagestyle{fancy}
\chead{\includegraphics[scale=0.2]{braian}}
\rfoot{\vspace{1cm} \includegraphics[scale=0.03]{Logo-Epitech}}

\newcommand{\HRule}{\rule{\linewidth}{0.5mm}}

\begin{document}
  \begin{titlepage}
  \begin{center}

    % Upper part of the page. The '~' is needed because \\
    % only works if a paragraph has started.
    \includegraphics[width=0.5\textwidth]{braian}~\\[1cm]

    \textsc{\LARGE Epitech Paris}\\[1.5cm]

    \textsc{\Large Fiche Projet - PFA}\\[0.5cm]

    % Title
    \HRule \\[0.4cm]
    { \huge \bfseries Brai(a)n - An AI who learn \\[0.4cm] }

    \HRule \\[1.5cm]

    % Author and supervisor
    \noindent

    \begin{flushleft} \large
      \emph{Chef Technique}\\
      Sebastien \textsc{Maire}
    \end{flushleft}

    \begin{flushright} \large
      \emph{Technique:} \\
      Alexandre \textsc{Fulgoni}
    \end{flushright}

    \begin{flushleft} \large
      \emph{Technique:} \\
      Stephane \textsc{Nguyen}
    \end{flushleft}

    \begin{flushright} \large
      \emph{Manager:} \\
      Thomas \textsc{Nieto}
    \end{flushright}

    \vfill

    % Bottom of the page
    {\large \today}

  \end{center}

\end{titlepage}



  \section*{Introduction}
  Au regard de ce que la science-fiction a imaginé dans le domaine de l'intelligence artificielle, l'informatique moderne n'en est qu'à ses balbutiements.
  Prenons quelques examples en littérature : HAL dans 2001, l'Odyssée de l'Espace ;
  en cinéma : JARVIS dans Iron-Man ou Mr. Smith dans Matrix ; ou encore dans le jeu-vidéo : Cortana dans HALO... Pour ne citer qu'eux. \\
  Le plus pertinant exemple d'IA qui nous accompagne ? Siri, une voix monotone dans un téléphone qui se décharge en dix heures...
  D'une part, nous avons de vraies personnages et d'autre part, une assistante électronique aux réponses hasardeuses.

  \noindent
  La différence la plus flagrante entre ces IA et Siri ? Leur comportement. \\
  En effet, le cadre de la science-fiction permet tous les possibles. Les IA qui en sont issues évoluent, intéragissent, échangent, se remettent en question.
  Seulement, cela n'est permis qu'avec la faculté d'appendre. Quid de l'apprentissage dans l'IA réelle ? \\

  \noindent
  C'est sur ce dernier point, que nous avons décidé de travailler. \\
  Nous nous concentrerons alors, sur l'aspect apprentissage et la prise de décision d'une intelligence artificielle.

  \section*{Equipe}
    \begin{description}
      \item[Sébastien Maire] \hfill \\
        \textbf{Chef Technique} \\
        Motivé par l'application de ses connaîssances en algorithmie et leur approndissement
      \item[Alexandre Fulgoni] \hfill \\
        \textbf{technique} \\
        A toujours été motivé par l'IA et leurs domaines d'application.
      \item[Stéphane Nguyen] \hfill \\
        \textbf{technique} \\
        intéressé par le développement de jeu-vidéo. En effet,
        au-delà du graphisme, l’IA est LE pillier qui permet le réalisme
        et donc l’immersion du joueur.
      \item[Emmanuel Isidore] \hfill \\
        \textbf{Recherche et technique} \\
       Après quelques essais dans l'IA, j'ai envie de pousser mes comp\'etences à leur limite sur un sujet qui m'attire
        Curieux de tout et motivé.
      \item[Thomas Nieto] \hfill \\
        \textbf{Recherche et management} \\
        Curieux de tout et motivé.
    \end{description}

  \noindent
  Première fois ensemble dans un groupe. Nous nous sommes rencontrés grâce aux
  intérêts que nous partageons et sommes donc tous motivés pour aller le plus
  loin possible.

  \section*{Contexte}
  Afin de tester les différentes versions de nos intelligences artificielles, nous allons nous servir d'environnements virtuels ouverts (dit sandbox).
  Ces derniers permettront de disposer de tous les éléments qui structurent notre monde, mais dans une version simpliste et programmable.
  Nous pensons, alors, utiliser les mécanismes de cycles de vies, de gravité, de faim, ainsi que les diverses interactions sociales, etc.

  \section*{Partenaire}
  L'Epitech Innovation HUB en la personne de Thibaut Royer.

  \section*{Objectifs}
  Dans un premier temps, l'\'equipe se consacrera à la recherche d'outils et de designs.
  Après quoi, nous serons en mesure de concevoir et d'impl\'ementer notre prorpre IA \'evolutive.
  Enfin, en fonction des recherches ant\'erieures et de la diffilcult\'ee choisie pour l'impl\'ementation, nous conceverons un outil de calibration, de design et de construction d'IA prenant example sur la notre.

  \section*{Planning}
  \begin{tabular}{c c|l}
    \hline
      d\'ebut & fin & \\
    \hline
      15 nov. & 15 dec. & Recherche \\
      15 dec. & 15 jan. & Conception IA \\
      15 jan. & 15 mar. & Impl\'ementation IA \\
      15 mar. & 30 mar. & Conception SDK\\
      30 mar. & fin & Impl\'ementation SDK\\
    \hline
  \end{tabular}
%   15 mar - 30 mar Conception SDK
%   30 mar - fin Implementation SDK

\end{document}
